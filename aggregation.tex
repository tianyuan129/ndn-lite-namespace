\section{Support Aggregation}
\label{sec:data-aggregation}

Once defined what elements should be included in ADU's name, a pratical issue emerges: in what order they are placed in name?
It is actually a highly applcation relevant question posed, given the way of ordering determines way of data aggregation happens.

The corelations of the four elements mentioned also complicate the decision. 
Notice that \textit{data type} can split \textit{capability} namespace into command and content, while \textit{capability} can also classify commands into different categorizes.
Same problem also exist for \textit{effective scope} and \textit{command identifier}.

We start order determining by making observation that in most cases \textit{effective scope} and \textit{command identifier} are finer granularity of \textit{data type} and \textit{capability}.
Hence the latter two should be at front of the former two.
Then we consider aggregation apporach can benefit application most. 

\begin{figure}[!h]
    \centering
    \includegraphics[width=0.5\textwidth]{comparison}
    \caption{Aggregation Approaches Comparison}
    \label{fig:aggregation-comparison}
\end{figure}

Another important observation made is \textit{capability} and \textit{effective scope} are more expressive and providing more data collection dimensions than the other two.
More expressive components should be at smaller depth inside the nametree, to avoid communication overhead of Pub/Sub.
An example is given in Fig~\ref{fig:aggregation-comparison} with three \textit{capability} are invovled. 
One have three dimensiosns of freedom to access namespace from \textsl{/<home-prefix>} when \textit{capability}'s position is upper than \textit{data type}.
In contrast, the other way of aggregation only provide two dimensions of freedom for namespace accessing at the entry point.
In the first organization, received data of subscription on the first depth are either commands or content.
However, data are less guessable when the same subscription happens on the second organization since \textit{capability} has numerous types and more expressive.
Same analysis applies on \textit{effective scope} and \textit{command identifier} situation, and \textit{effective scope} prevails in our application scenarios.

Finally we get the total order benefits aggregation most: \textsl{/<home-prefix>/<capability>/<data-type>/<effective-scope>/<data-id>}.
This order also better fit the functionality separation feature in Section~\ref{sec:separation}.
We argue this order is the outcome of current NDN-Lite applcation needs of accessing namespace with maximized freedom, and this order may change if applcation needs shift in future.