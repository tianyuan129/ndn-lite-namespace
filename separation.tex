\section{Separate Functionalities}
\label{sec:separation}

In order to build the best possible system, an NDN-based IoT system should consider both the basic NDN features and common application and device features.
We first consider what functionalities such a home IoT system should provide. 

To serve as a basic NDN entity, one should have four pieces: Name, Certificate, Trust Anchor, and Trust Schema.
We claimed that since they're all first delivered or installed at entities' security bootstrapping time, these trust management should remain in the same sub-namespace.
Then service discovery and access control are needed for the system to faciliate application interaction and content protection, corresponding sub-namespace are needed. 
Considering the independency of the above three functionalities, they should all be kept into non-overlapping subnamespace to avoid conflicts.

Besides basic bricks for a NDN IoT system, common application or devices features should also be considered.
In current smart home systems devices are often categorized into different capabilities. Applications are represented as consumers for this device provided capabilities.
This style of categorization benefit platform-level Pub/Sub much and become an emerging trend of current IoT systems.
NDN-Lite also follows a similar approach to bring entity's capabilities directly under the root prefix (e.g., \textsl{/alice-home}) to better serve our Pub/Sub~\ref{sec:data-aggregation}.

To summarize, system-level functionalities are separated into different subnamespaces to ease both trust management and data aggregation.
Therefore, NDN-Lite separates overall namespace into following subnamespaces:

\begin{itemize}
\setlength{\itemsep}{0pt}
    \item \textsl{/<home-prefix>/cert}
    \item \textsl{/<home-prefix>/sd}
    \item \textsl{/<home-prefix>/ac}
    \item \textsl{/<home-prefix>/<capability>}
\end{itemize}